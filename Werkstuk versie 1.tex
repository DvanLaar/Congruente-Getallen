\documentclass[12pt,reqno]{article}
\usepackage[dutch]{babel}
\usepackage[margin=1.2in]{geometry}
\usepackage{amsmath}
\usepackage{amssymb}
\usepackage{amsthm}
\usepackage{natbib}
\usepackage{tikz}
\usepackage{url}

\title{\textbf{Congruente getallen}\\
		\small{eerste versie studentencheck}}
\author{
	\begin{tabular}{ l l }
		Lotte Bruijnen, & 4297652 \\
		Daan van Laar, & 5518741 \\
		Suzanne Vincken, & 4273338
	\end{tabular}\\
	Onder begeleiding van: Carel Faber, UU
}
\date{10-12-2015}

\newcommand*{\NN}{\ensuremath{\mathbb{N}}}
\newcommand*{\ZZ}{\ensuremath{\mathbb{Z}}}
\newcommand*{\QQ}{\ensuremath{\mathbb{Q}}}
\newcommand*{\RR}{\ensuremath{\mathbb{R}}}
\newcommand*{\CC}{\ensuremath{\mathbb{C}}}
\newcommand*{\QED}{\hfill\ensuremath{\blacksquare}}

\begin{document}
	
	\maketitle
	%\thispagestyle{empty}
	%\pagestyle{empty}
	\allowdisplaybreaks
	
	\section{Inleiding}
	
	\section{Stelligen}
	
	\newtheorem{Tunnell}{Theorem}
	\begin{Tunnell}
		\text{\cite{oort_p3}} (Tunnell, 1983) If $n$ is a squarefree and odd (respectively, even) positive integer ans $n$ is the area of a right triangle with rotional sides, then
		\begin{align}
		\#\{x,y,z\in\ZZ|n=2x^2+y^2+32z^2\} = \frac{1}{2}\#\{x,y,z\in\ZZ|n=2x^2+y^2+8z^2\}\\
		\notag \text{(respectively, } \\
		\#\{x,y,z\in\ZZ|\frac{n}{2}=4x^2+y^2+32z^2\} = \frac{1}{2}\#\{x,y,z\in\ZZ|\frac{n}{2}=4x^2+y^2+8z^2\})
		\end{align}
		If the weak Birch-Swinnerton-Dyer conjecture is true for the elliptic curves $E_n:y^2=x^3-n^2x$, then, conversely, these equalities imply that $n$ is a congruent number.
	\end{Tunnell}
	
	\newtheorem{CG}{Defenitie}
	\begin{CG}
		Een positief geheel getal $N$ heet een congruent getal als er een rechthoekige driehoek bestaat met lengtes van zijden in $\QQ_{>0}$ en met oppervlak gelijk aan $N\in\ZZ$. Noem de lengtes van de zijden $a,b,c\in\QQ$; met behulp van de stelling van Pythagoras zien we:
		
	\end{CG}
	
	
	\bibliographystyle{plain}
	\bibliography{bib}
	
\end{document}